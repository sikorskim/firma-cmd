
    \documentclass[a4paper,10pt]{article}
    \usepackage{latexsym}
    \usepackage[polish]{babel}
    \usepackage[utf8]{inputenc}
    \usepackage[MeX]{polski}
    \usepackage{float}
    \usepackage{geometry}
    \usepackage[table]{xcolor}

    \geometry{
    a4paper,
    total={170mm,257mm},
    left=20mm,
    top=20mm,
    }

    \pagenumbering{gobble}
    \date{}

    \begin{document}

    \noindent
    \textbf{{0}}\\
    {1}\\
    NIP {2} REGON {3}\\
    Tel. {4}\\
    E-mail: {5}\\
    {6}\\
    \\
    {7}
    {8}\\
    \\
  
\noindent
    \textbf{Data i miejsce wystawienia}		\hfill \textbf{Data dostawy/wykonania usługi}\\
    {0}, {1}							        \hfill	{2}\\
  
    \begin{center}
    \textbf{\huge{FAKTURA FV/2018/6/1}}
    \end{center}
    \vspace{2cm}
  
    \noindent
    \textbf{SPRZEDAWCA}						\hfill \textbf{NABYWCA}\\
    {0}					                        \hfill {3}\\
    {1}								            \hfill {4}\\
    NIP {2}								        \hfill NIP {5}\\
  
    \begin{table}[H]
    \raggedleft
    \begin{tabular}{| p{0,5cm} | p{4,8cm}  | p{0,7cm} | p{0,7cm}| p{1,3cm} | p{1,3cm} | p{1,1cm} | p{1,3cm} | p{1,3cm} |}
    \hline
    \textbf{L.p.} & \textbf{Nazwa} & \textbf{J.m.} & \textbf{Ilość} & \textbf{Cena netto} & \textbf{Wartość netto} & \textbf{Stawka VAT} & \textbf{Wartość VAT} & \textbf{Wartość brutto} \\ \hline
  
    \hline
    \multicolumn{5}{|r|}{\textbf{Razem}} & \textbf{{0}} & \cellcolor[gray]{0.9} & \textbf{{1}} & \textbf{{2}} \\ \hline
    \end{tabular}
    \end{table}
  
    \begin{table}[H]
    \raggedleft
    \begin{tabular}{| p{2,5cm} | p{1,3cm} | p{1,1cm} | p{1,3cm} | p{1,3cm} |}
    \hline
    \textbf{Według stawki VAT} & \textbf{Wartość netto} & \cellcolor[gray]{0.9} & \textbf{Kwota VAT} & \textbf{Wartość brutto} \\ \hline
  
    {0}\% & {1} & \cellcolor[gray]{0.9} & {2} & {3} \\ \hline
  
  \hline
  {\textbf{Razem}}  & \textbf{100,00} & \cellcolor[gray]{0.9} & \textbf{23,00} & \textbf{123,00} \\ \hline
  \end{tabular}
  \end{table}

  \noindent
  \begin{flushright}
  \textbf{\LARGE{Do zapłaty} {0} PLN}\\
  {1} PLN {2}/100
  \end{flushright}

  \vfill
  \noindent
  \textbf{Forma płatności}\\
  {0}\\
  \textbf{Termin płatności}\\
  {1}\\
  \\
  
    \textbf{Fakturę wystawił}\\
    {0}
    \end{document}
  